\documentclass[british]{article}

\usepackage[british]{babel}% Recommended
\usepackage{csquotes}% Recommended

\usepackage[sorting=nyt,style=apa]{biblatex}

\addbibresource{~/Tex/library.bib}

\usepackage[margin=1in]{geometry}

\usepackage{amsmath}
\usepackage{graphicx}
\usepackage{listings}
\usepackage{enumerate}
\newcommand{\code}[1]{\texttt{#1}}
\newtheorem{defin}{Definition}
\newtheorem{prop}{Proposition}
\newtheorem{col}{Corollary}
\newtheorem{thm}{Theorem}
\setlength{\parskip}{1em}


\title{CS4402 - Practical 1: Bombastic}
\author{170008773}
\date{\today}
\begin{document}
\maketitle


\begin{abstract}
 
\end{abstract}
{\bf Keywords:} 



\section{Introduction}
\label{intro}

 
 
\section{Problem description}
\label{problem}
\subsection{The game}
Bombastic is played on a $N\times M$ grid of cells. The cells in this grid can either be \code{dead}, \code{ice}, or \code{normal}. \code{dead} cells cannot be entered by anything, and other cells can accomodate at most one block or the avatar. For simplicities sake we assume that every grid is surrounded by a wall of dead cells. On this grid there is an avatar and one or more blocks and a number of goals, \texttt{equal to the number of blocks}. The objective of the avatar is to walk around the gird and push the blocks around until all blocks are at a goal. In this scenario we are not interested in which block ends up at which goals.  


\subsection{Avatar logic}
The avatar is allowed to move around the grid, moving the cell it is currently occupying to any of the adjacent cells that are not dead. The avatar is only allowed to move purely horizontally or vertically (i.e. not at the time) and not moving is also disallowed.  The player is allowed to move onto an ice cell, however when it moves off that cell again, the ice will crack, turning the cell into a dead one. 


\subsection{Box logic}
The avatar can move blocks by moving into their square. This will move the block one square in the direction the avatar is travelling in. This is only allowed if the cell the block is moving into is not dead and does not contain another block.

\subsection{Objectives}
Given the grid layout, the positions of the blocks, the position of the goals, and the initial position of the avatar, the objective is to find a sequence of legal moves that move all the blocks to a goal. In this instance the number of moves is provided. So the problem is to find whether there is a sequence of the given length that satisfies all the objectives. 


\section{Modeling the problem}
\label{model}
\subsection{Setup}
In this instance I was required to use the modelling language \code{Essence prime} in conjunction with the constraint solver Savile Row. I was also provided with the decision variables, their domains and several sets of parameters to test the system.   

\subsection{Designed instances}
I was also required to design new instances of this problem class. I took this opportunity to design a few instances that can be used to either test specific parts of the modelled logic or some artificially difficult problem to test the performance. They are designed in pairs. Every problem has one paramter file that is solvable and one that isn't so that we can see how the performance compares to comperable problems. We will discuss them below. Human readable maps of all the instances are provided in the files. 
\begin{enumerate}[10)]
	\item This problem is a slight adaptation of the \code{Bombastic1\_1.param}. It is just a single width corridor with two blocks in it. This is designed to test the avatar's inability to push more than one block at the same time.
\end{enumerate}
	\paragraph{2Blocks} This problem is a slight adaptation of the \code{Bombastic1\_1.param}. It is just a single width corridor with two blocks in it. This is designed to test the avatar's inability to push more than one block at the same time. 
	\paragraph{BigComplex} This instance is a simple adaptation of \code{Bombastic9\_17.param}.  It is identical to the old instance but it has a relatively big empty space added at the bottom that should change nothing about the solution. This was done to test how well the solver would fair if a lot of useless space was added. 
	\paragraph{iceT} This problem contains a square of ice in the middle with two single width corridors with a block in them on either end. The possible version has one ice cell replaced with a normal one, making it possible. This instance was designed to test how the solver deals with ice planning, and whether the ice mechanic is done correctly. The impossible version also has a step length less than the possible version. This is not strictly necessary since the impossibility comes form the ice configuration, but this made our naming scheme and subsequently the data collection and visualisation much easier, while it shouldn't change the final outcome. 
	\paragraph{L} This is another instance designed to test the ice mechanic. There is a L shaped corridor with an ice cell at the intersection that the avatar starts on and a block and goal in each branch. The possible version removes the ice cell. The impossible version also has a step length less than the possible version. This is not strictly necessary since the impossibility comes form the ice configuration, but this made our naming scheme and subsequently the data collection and visualisation much easier, while it shouldn't change the final outcome.  
	\paragraph{T} This instance is a slightly more complicated instance to test the inability of the avatar to push more than one box at a time. It has a small room and two blocks in a row with goals on two sides. This instance requrest a tiny bit of planning, since the avatar will have to walk around the blocks first. 
	\paragraph{WideOpen} This instance is simply a large open room with no complications. It is designed to test how the solver performs in terms of the length of the solutions and the size of the possible moves while the actual solution is very uncomplicated. 

\section{Emperical evluation}
\label{results}

\section{Conclusion}
\label{conclusion}

 
 
 
word count: 
\printbibliography
\end{document}
